%%%%%%%%%%%%%%%%%%%%%%%%%%%%%%%%%%%%%%%%%%%%%%
% An example of a lab report write-up.
%%%%%%%%%%%%%%%%%%%%%%%%%%%%%%%%%%%%%%%%%%%%%%
% This is a combination of several labs that I have done in the past for
% Computer Engineering, so it is not to be taken literally, but instead used as
% a great starting template for your own lab write up.  When creating this
% template, I tried to keep in mind all of the functions and functionality of
% LaTeX that I spent a lot of time researching and using in my lab reports and
% include them here so that it is fairly easy for students first learning LaTeX
% to jump on in and get immediate results.  However, I do assume that the
% person using this guide has already created at least a "Hello World" PDF
% document using LaTeX (which means it's installed and ready to go).
%
% My preference for developing in LaTeX is to use the LaTeX Plugin for gedit in
% Linux.  There are others for Mac and Windows as well (particularly MikTeX).
% Another excellent plugin is the Calc2LaTeX plugin for the OpenOffice suite.
% It makes it very easy to create a large table very quickly.
%
% Professors have different tastes for how they want the lab write-ups done, so
% check with the section layout for your class and create a template file for
% each class (my recommendation).
%
% Also, there is a list of common commands at the bottom of this document.  Use
% these as a quick reference.  If you'd like more, you can view the "LaTeX Cheat
% Sheet.pdf" included with this template material.
%
% (c) 2009 Derek R. Hildreth <derek@derekhildreth.com> http://www.derekhildreth.com
% This work is licensed under the Creative Commons Attribution-NonCommercial-ShareAlike License. To view a copy of this license, visit http://creativecommons.org/licenses/by-nc-sa/1.0/ or send a letter to Creative Commons, 559 Nathan Abbott Way, Stanford, California 94305, USA.
%%%%%%%%%%%%%%%%%%%%%%%%%%%%%%%%%%%%%%%%%%%%%%
\documentclass[aps,letterpaper,10pt]{revtex4}
\input kvmacros % For Karnaugh Maps (K-Maps)

\usepackage{graphicx} % For images
\usepackage{float}    % For tables and other floats
\usepackage{verbatim} % For comments and other
\usepackage{amsmath}  % For math
\usepackage{amssymb}  % For more math
\usepackage{fullpage} % Set margins and place page numbers at bottom center
\usepackage{listings} % For source code
\usepackage{subfig}   % For subfigures
\usepackage[usenames,dvipsnames]{color} % For colors and names
\usepackage[pdftex]{hyperref}           % For hyperlinks and indexing the PDF
\hypersetup{ % play with the different link colors here
    colorlinks,
    citecolor=blue,
    filecolor=blue,
    linkcolor=blue,
    urlcolor=blue % set to black to prevent printing blue links
}

\definecolor{mygrey}{gray}{.96} % Light Grey
\definecolor{mygreen}{rgb}{0,0.6,0}
\definecolor{mygray}{rgb}{0.5,0.5,0.5}
\definecolor{mymauve}{rgb}{0.58,0,0.82}
\definecolor{myblue}{rgb}{0.13,0.13,1}

% 默认的C++配置
\lstset{
	language=[ISO]C++,              % choose the language of the code ("language=Verilog" is popular as well)
   tabsize=3,							  % sets the size of the tabs in spaces (1 Tab is replaced with 3 spaces)
	basicstyle=\tiny,               % the size of the fonts that are used for the code
	numbers=left,                   % where to put the line-numbers
	numberstyle=\tiny,              % the size of the fonts that are used for the line-numbers
	stepnumber=2,                   % the step between two line-numbers. If it's 1 each line will be numbered
	numbersep=5pt,                  % how far the line-numbers are from the code
	backgroundcolor=\color{mygrey}, % choose the background color. You must add \usepackage{color}
	%showspaces=false,              % show spaces adding particular underscores
	%showstringspaces=false,        % underline spaces within strings
	%showtabs=false,                % show tabs within strings adding particular underscores
	frame=single,	                 % adds a frame around the code
	tabsize=3,	                    % sets default tabsize to 2 spaces
	captionpos=b,                   % sets the caption-position to bottom
	breaklines=true,                % sets automatic line breaking
	breakatwhitespace=false,        % sets if automatic breaks should only happen at whitespace
	%escapeinside={\%*}{*)},        % if you want to add a comment within your code
	commentstyle=\color{BrickRed}   % sets the comment style
}

% 专门为Python优化的配置
\lstdefinestyle{pythonstyle}{
    language=Python,
    basicstyle=\small\ttfamily,
    backgroundcolor=\color{mygrey},
    commentstyle=\color{mygreen},
    keywordstyle=\color{blue},
    numberstyle=\tiny\color{mygray},
    stringstyle=\color{mymauve},
    identifierstyle=\color{black},
    showstringspaces=false,
    numbers=left,
    numbersep=8pt,
    stepnumber=1,
    frame=single,
    rulecolor=\color{black},
    breaklines=true,
    breakatwhitespace=true,
    postbreak=\mbox{\textcolor{red}{$\hookrightarrow$}\space},
    tabsize=4,
    captionpos=b,
    morekeywords={self, True, False, None, as, with, yield, async, await},
    deletekeywords={format},
    sensitive=true,
    morecomment=[l]{\#},
    morestring=[b]',
    morestring=[b]",
    morestring=[s]{'''}{'''},
    morestring=[s]{"""}{"""},
}

% Make units a little nicer looking and faster to type
\newcommand{\Hz}{\textsl{Hz}}
\newcommand{\KHz}{\textsl{KHz}}
\newcommand{\MHz}{\textsl{MHz}}
\newcommand{\GHz}{\textsl{GHz}}
\newcommand{\ns}{\textsl{ns}}
\newcommand{\ms}{\textsl{ms}}
\newcommand{\s}{\textsl{s}}



% TITLE PAGE CONTENT %%%%%%%%%%%%%%%%%%%%%%%%
% Remember to fill this section out for each
% lab write-up.
%%%%%%%%%%%%%%%%%%%%%%%%%%%%%%%%%%%%%%%%%%%%%
\newcommand{\labno}{05}
\newcommand{\labtitle}{AI 3603 Artificial Intelligence: Principles and Techniques}
\newcommand{\authorname}{Chen Jingtao \left(523030910028\right)}
\newcommand{\hw}{1}
% END TITLE PAGE CONTENT %%%%%%%%%%%%%%%%%%%%


\begin{document}  % START THE DOCUMENT!


% TITLE PAGE %%%%%%%%%%%%%%%%%%%%%%%%%%%%%%%%%%%%%%
% If you'd like to change the content of this,
% do it in the "TITLE PAGE CONTENT" directly above
% this message
%%%%%%%%%%%%%%%%%%%%%%%%%%%%%%%%%%%%%%%%%%%%%%%%%%%
\begin{titlepage}
\begin{center}
{\Large \textsc{\labtitle} \\ \vspace{4pt}}
\rule[13pt]{\textwidth}{1pt} \\ \vspace{150pt}
{\large By: \authorname \\ \vspace{10pt}
HW\#: \hw \\ \vspace{10pt}
\today}
\end{center}
\end{titlepage}
% END TITLE PAGE %%%%%%%%%%%%%%%%%%%%%%%%%%%%%%%%%%





%%%%%%%%%%%%%%%%%%%%%%%%%%%%%%
%%%%%%%%%%%%%%%%%%%%%%%%%%%%%%
\section{Introduction}
%No Text Here
%%%%%%%%%%%%%%%%%%%%%%%%%%%%%%%

This Lab aims to design an algorithm to guide from the start point to the target point avoiding obstacles based on the A* algorithm. The algorithm should be implemented in Python language. And it's required to generate a smooth trajectory based on the path found by the A* algorithm using cubic spline interpolation.\vspace{3mm}

The Lab is divided into three steps:\vspace{3mm}

\begin{itemize}
	\item Implement basic A* algorithm to find a path from start to target avoiding obstacles
	\item Optimize the A* algorithm to achieve a better path
	\item Generate a smooth path based on the path found by the A* algorithm 
\end{itemize}
\vspace{3mm}

%%%%%%%%%%%%%%%%%%%%%%%%%%%%%%
%%%%%%%%%%%%%%%%%%%%%%%%%%%%%%
\section{Tasks}

\subsection{Task 1: Implement basic A* algorithm to find a path from start to target avoiding obstacles}

The first task is to implement the basic A* algorithm to find a path from start to target avoiding obstacles. The A* algorithm is a popular pathfinding and graph traversal algorithm that is used in many applications, including games and robotics. The algorithm uses a heuristic function to estimate the cost of reaching the target from the current node, and it combines this with the cost of reaching the current node from the start node to determine the total cost of each node. The algorithm then explores the nodes with the lowest total cost first, until it reaches the target node.\vspace{3mm}

As for the search strategy we pick up BFS (Breadth-First Search) to implement the A* algorithm. \vspace{3mm}

The heuristiuc function we choose is the Euclidean distance between the current node and the target node. And the cost of reaching the current node from the start node is simply the number of steps taken to reach that node.

\begin{itemize}
	\item $f\left(x\right) = g\left(x\right) + h\left(x\right)$
	\item $g\left(x\right) = $ cost from start to current node
	\item $h\left(x\right) = $ Euclidean distance from current node to target
\end{itemize}
	% You can refer to this set of images by using \ref{fig:oscil}.  ie "please refer to Figure \ref{fig:oscil}."
	% You can refer to a specific subimage by using \ref{fig:Per6A}. ie "please refer to Figure \ref{fig:Per6A}."
   % I prefer the quality of a .png image, but you may use other extensions such as .jpg.
	% \begin{figure}[H]
	%   \centering
	%   \subfloat[LED4 Period]{\label{fig:Per6A}\includegraphics[width=0.4\textwidth]{period_led4.png}} \\
	%   \subfloat[LED5 Period]{\label{fig:Per6A}\includegraphics[width=0.4\textwidth]{period_led5.png}}
	%   \subfloat[LED6 Period]{\label{fig:Per6A}\includegraphics[width=0.4\textwidth]{period_led6.png}}
	%   \caption{Period of LED blink rate captured by osciliscope.}
	%   \label{fig:oscil}
	% \end{figure}

%%%%%%%%%%%%%%%%%%%%%%%%%%%%%%
%%%%%%%%%%%%%%%%%%%%%%%%%%%%%%

The code implementation is as follow:\vspace{5mm}

\lstinputlisting[style=pythonstyle]{5-Task_1.py}
\vspace{3mm}

In the front of the code we implement the Queue and its descendant PriorityQueue to support the A* algorithm. Then we implement the AStarPlanner function to encapsulate the A* algorithm. The code here are reused from another project I did before.
\vspace{3mm}

The main function is the plan function which takes in the start and target positions and returns the path found by the algorithm. The function uses a priority queue to explore the nodes with the lowest total cost first, and it keeps track of the visited nodes to avoid cycles using a dictionary bestg. 

\vspace{3mm}

Finally we get a path as the graph shown below:

\vspace{3mm}

\begin{figure}[H]
   \begin{center}
	  \includegraphics[width=0.6\textwidth]{5-Task1-Path.png}
   \end{center}
\end{figure}

% IF YOU'D RATHER TYPE THE CODE, OR HAVE A SMALLER BLOCK OF CODE, USE THIS:
%\begin{lstlisting}
%if(something)
%	do this
%else
%	do this
%\end{lstlisting}

%% THIS IS FROM A DIFFERENT CLASS, BUT DEMONSTRATES MATH MODE WELL
%%%%%%%%%%%%%%%%%%%%%%%%%%%%%%
\subsection{Task 2: Optimize the A* algorithm to achieve a better path}

Basing on Task-1, the assignment require to optimize our path according to following three rules:

\vspace{3mm}

\begin{itemize}
	\item Be able to turn in 8 directions (N, NE, E, SE, S, SW, W, NW)
	\item Consider the distance between the current node and the wall
	\item Consider the cost of steering
\end{itemize}

Our algorithm framework is similar to Task-1, and we make some modifications to the searching-part and the heuristic function.

\vspace{3mm}

\lstinputlisting[style=pythonstyle]{6-Task_2.py}

\vspace{3mm}

When generating the $(dx, dy)$ pairs, we consider 8 directions instead of 4 directions in Task-1.

\vspace{3mm}

When calculating the heruistic function, we add two more factors: The shortest distance between the cureent node and the wall, and the cost of steering.

\vspace{3mm}

The shortest distance between the current node and the wall is calculated in a new function. It will be a negative value to the h-function since we want our robot to stay away from the wall.

\vspace{3mm}

The cost of steering is calculated only if our current direction is different from the last direction. To implement this, we judge if we are changing direction and pass it to our h-function in the searching-part.

\vspace{3mm}

At last we assign a proper weight to each factor in the h-function to get a better path.

\vspace{3mm}

\begin{gather}
	h\left(x\right) = \text{EuclideanDistance} - 3 \cdot \text{DistanceToWall} + 10 \cdot \text{SteeringCost} \\
	\text{SteeringCost} = \begin{cases}
		1, & \text{if changing direction} \\
		0, & \text{otherwise}
	\end{cases}
\end{gather}

\vspace{3mm}

And here is the result path that:

\vspace{3mm}

\begin{figure}[H]
   \begin{center}
	  \includegraphics[width=0.6\textwidth]{6-Task2-Path.png}
   \end{center}
\end{figure}

Additionally, we compare our A-star algorithm in Task-1 and Task-2. For example, we calculate the time consumption of both Task-1 and Task-2. The result is as follow:

\begin{itemize}
	\item Time consumption of Task-1: 0.231 s
	\item Time consumption of Task-2: 0.732 s
\end{itemize}

Task-2 apparently cost more time than Task-1, which is reasonable since we add more factors in the heuristic function and expand our searching directions, so our searching queue is larger than before.

\vspace{3mm}

And practically, the path in Task-2 is better than Task-1, since it avoids obstacles better. If we follow the path we found in Task-1, which is too close to the wall, our robot may collide with the wall due to some errors in movement. In Task-2, we consider the distance to the wall, so our robot can avoid obstacles better. Besides, we move towards eight directions instead of four, which gives us more flexibility in navigating the environment.

\vspace{3mm}

So in all, though Task-2 costs more time, it gives us a better path to avoid obstacles.

%%%%%%%%%%%%%%%%%%%%%%%%%%%%%%
\subsection{Task 3: Generate a smooth path based on the path found by the A* algorithm}

To generate a smooth path, the assignment offer me three methods, and I choose to dig into the Polynomial Interpolation, not of any reasons, only because it seems most easy, lol.

\vspace{3mm}

My strategy is brutely simple:

\begin{itemize}
	\item First, use our optimized A* algorithm in Task-2 to find the initial path
	\item Then, simplify the path by only recording the steering nodes.
	\item Finally, we use cubic spline to link every two neighbor nodes to form an overall smooth path, and sampling nodes equidistantly on the smooth path to plot our final path.(Here sampling 100 nodes)
\end{itemize}

To achieve the cubic spline interpolation, I use the CubicSpline function in the scipy.interpolate library.

The code implementation is as follow:

\vspace{5mm}

\lstinputlisting[style=pythonstyle]{7-Task_3.py}

\vspace{3mm}

The result we get is quite satisfying:

\begin{figure}[H]
   \begin{center}
	  \includegraphics[width=0.6\textwidth]{7-Task3-Path.png}
   \end{center}
\end{figure}

\vspace{3mm}

It is more similar to the real-world robot path-planning, and it avoids sudden turns, which is better for the robot's movement.

\section{Conclusion}

In this report, we have explored the implementation of an optimized A* algorithm for robot path planning. We compared the performance of the algorithm in two different tasks, highlighting the improvements made in Task-2 by considering additional factors in the heuristic function and expanding the search directions. Although Task-2 required more computational time, it resulted in a more effective path that better avoided obstacles.

\vspace{3mm}

Furthermore, we applied cubic spline interpolation to smooth the path generated by the A* algorithm in Task-2. The resulting smooth path demonstrated improved characteristics for real-world robot navigation, minimizing sudden turns and enhancing overall movement efficiency.

\vspace{3mm}

In conclusion, the enhancements made in the A* algorithm and the subsequent path smoothing techniques have significantly contributed to the development of a more robust and effective robot path-planning solution.

\vspace{3mm}

And from this lab, I have gained a deeper understanding of path-planning algorithms and their practical applications in robotics. The experience has been invaluable in enhancing my skills in algorithm design and implementation.

\vspace{3mm}

Thanks for your patience in reading my lab report!

\end{document} % DONE WITH DOCUMENT!


%%%%%%%%%%
PERSONAL FAVORITE LAB WRITE-UP STRUCTURE
%%%%%%%%%%
\section{Introduction}
	% No Text Here
	\subsection{Purpose}
		% Lab objective
	\subsection{Equipment}
		% Any and all equipment used (specific!)
	\subsection{Procedure}
		% Overview of the procedure taken (not-so-specific!)
\newpage
\section{Schematic Diagrams}
	% Any schematics, screenshots, block
   % diagrams used.  Possibly photos or
	% images could go here as well.
\newpage
\section{Experiment Data}
	% Depending on lab, program code would be
	% included here without the Estimated and
	% Actual Results.
	\subsection{Estimated Results}
		% Calculated. What it should be.
	\subsection{Actual Results}
		% Measured.  What it actually was.
\newpage
\section{Discussion \& Conclusion}
	% 3 Paragraphs:
		% Restate the objective of the lab
		% Discuss personal trials, errors, and difficulties
		% Conclude the lab


%%%%%%%%%%%%%%%%
COMMON COMMANDS:
%%%%%%%%%%%%%%%%
% IMAGES
begin{figure}[H]
   \begin{center}
      \includegraphics[width=0.6\textwidth]{RTL_SCHEM.png}
   \end{center}
\caption{A screenshot of the RTL Schematics produced from the Verilog code.}
\label{RTL}
\end{figure}

% SUBFIGURES IMAGES
\begin{figure}[H]
  \centering
  \subfloat[LED4 Period]{\label{fig:Per4}\includegraphics[width=0.4\textwidth]{period_led4.png}} \\
  \subfloat[LED5 Period]{\label{fig:Per5}\includegraphics[width=0.4\textwidth]{period_led5.png}}
  \subfloat[LED6 Period]{\label{fig:Per6}\includegraphics[width=0.4\textwidth]{period_led6.png}}
  \caption{Period of LED blink rate captured by osciliscope.}
  \label{fig:oscil}
\end{figure}

% INSERT SOURCE CODE
% For C++ code (default style)
\lstset{language=Verilog, tabsize=3, backgroundcolor=\color{mygrey}, basicstyle=\small, commentstyle=\color{BrickRed}}
\lstinputlisting{MODULE.v}

% For Python code (beautiful style)
\lstinputlisting[style=pythonstyle]{your_python_file.py}

% Or for inline Python code
\begin{lstlisting}[style=pythonstyle]
def hello_world():
    print("Hello, World!")
    return True
\end{lstlisting}

% TEXT TABLE
\begin{table}
\begin{center}
\begin{tabular}{|l|c|c|l|}
	x & x & x & x \\ \hline
	x & x & x & x \\
	x & x & x & x \\ \hline
\end{tabular}
\caption{Caption}
\label{label}
\end{center}
\end{table}

% MATHMATICAL ENVIRONMENT
$ 8 = 2 \times 4 $

% CENTERED FORMULA
\[  \]

% NUMBERED EQUATION
\begin{equation}
	
\end{equation}

% ARRAY OF EQUATIONS (The splat supresses the numbering)
\begin{align*}
	
\end{align*}

% NUMBERED ARRAY OF EQUATIONS
\begin{align}
	
\end{align}

% ACCENTS
\dot{x} % dot
\ddot{x} % double dot
\bar{x} % bar
\tilde{x} % tilde
\vec{x} % vector
\hat{x} % hat
\acute{x} % acute
\grave{x} % grave
\breve{x} % breve
\check{x} % dot (cowboy hat)

% FONTS
\mathrm{text} % roman
\mathsf{text} % sans serif
\mathtt{text} % Typewriter
\mathbb{text} % Blackboard bold
\mathcal{text} % Caligraphy
\mathfrak{text} % Fraktur

\textbf{text} % bold
\textit{text} % italic
\textsl{text} % slanted
\textsc{text} % small caps
\texttt{text} % typewriter
\underline{text} % underline
\emph{text} % emphasized

\begin{tiny}text\end{tiny} % Tiny
\begin{scriptsize}text\end{scriptsize} % Script Size
\begin{footnotesize}text\end{footnotesize} % Footnote Size
\begin{small}text\end{small} % Small
\begin{normalsize}text\end{normalsize} % Normal Size
\begin{large}text\end{large} % Large
\begin{Large}text\end{Large} % Larger
\begin{LARGE}text\end{LARGE} % Very Large
\begin{huge}text\end{huge}   % Huge
\begin{Huge}text\end{Huge}   % Very Huge


% GENERATE TABLE OF CONTENTS AND/OR TABLE OF FIGURES
% These seem to have some issues with the "revtex4" document class.  To use, change
% the very first line of this document to "article" like this:
% \documentclass[aps,letterpaper,10pt]{article}
\tableofcontents
\listoffigures
\listoftables

% INCLUDE A HYPERLINK OR URL
\url{http://www.derekhildreth.com}
\href{http://www.derekhildreth.com}{Derek Hildreth's Website}

% FOR MORE, REFER TO THE "LINUX CHEAT SHEET.PDF" FILE INCLUDED!
